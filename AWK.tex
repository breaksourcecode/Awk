\documentclass[10pt,fancyhdr,UTF-8]{ctexart}
\setmainfont{Times New Roman}
\setCJKmainfont{SimSun}
\setCJKfamilyfont{song}{STSong}
\newcommand\zh[1]{{\CJKfamily{song}#1}}
%\usepackage[centering,paperwidth=180mm,paperheight=230mm,body={390pt,530pt},showframe]{geometry}
\usepackage[centering,body={390pt,530pt},showframe]{geometry}
\usepackage{latexsym,amssymb,amsmath}
%\usepackage[top=0in,bottom=10in,left=1.25in,right=1.25in]{geometry}
\geometry{left=2.5cm,right=2.5cm,bottom=12.5cm,top=0cm}
\usepackage[a4,center,frame,color=blue]{crop}
\usepackage{indentfirst} %首行缩进
\setlength{\parindent}{2em}
\usepackage{titlesec}
\usepackage{CJK}
%\usepackage{titling}
%\pretitle{\begin{flushright}\large}
\usepackage[flushleft]{caption}
\usepackage{fancyhdr}
\pagestyle{fancy}
\rfoot{\thepage}
%\fancyhf{}
%\fancyhead[RE]{\normalfont\small\rmfamily\nouppercase{\leftmark}}
%\fancyhead[LO]{\normalfont\small\rmfamily\nouppercase{\rightmark}}
%\fancyhead[LE,RO]{\thepage}
%\fancyhead[LE,LO]{\small Book Title}
\topmargin=0pt
\oddsidemargin=0pt \evensidemargin=0pt
\textwidth=14cm
\textheight=21cm
\title{AWK教程}
\author{half-beast@163.com}
\date{Jun. 1 , 2015}
\begin{document}
\maketitle
\section{AWK教程}
\indent Awk是一门方便其表达力丰富的编程语言,它适用于多变量计算和数据操作任务。本章是一个教程,目标是使你能够尽快的用awk写出属于自己的程序。第二章详细描述了语言的方方面面,而剩下的章节则向你展示awk习惯于解决许多领域的问题。整本书中,我们都试图挑选一些让你感觉既有意思同时又很有启发性的例子。\par
\subsection{开始}
\indent 有用的awk程序通常都很短,有的甚至只有一到两行。假设你有一份称为emp.data的文件。它包含雇员名字,时新和雇员的工作时常,一条雇员记录单独成行,如下所示:\\
\indent Beth 4.00 0\\
\indent Dan  3.75 0\\
\indent Kathy 4.00 10\\
\indent Mark 5.00 20\\
\indent Mary 5.50 22\\
\indent Susie 4.25 18\\
现在你想打印所有那些工作时间超过零小时的雇员名称和个人所得。awk天生擅长这类工作,对awk来说,这太容易了。只需敲入以下命令行:\\
awk `\$3 > 0 \{ print \$1 , \$2  *  \$3  \}' emp.data \\
你应该得到如下输出:\\
\indent Kathy   40\\
\indent Mark    100\\
\indent Mary    110\\
\indent Susie   76.5\\
\indent 这条命令告诉系统使用awk来执行单引号内的程序,并从输入文件emp.data中获取处理数据。单引号中的内容是一条完整的awk程序。它由单条pattern-action语句构成。\\
pattern--\$3 > 0,用于匹配每一输入行的第三列或字段是否大于0,对应的action--\{ print \$1 , \$2 * \$3  \}打印匹配行的第一列和第二列与第三列的乘积。\\
如果你想打印那些没有工作的雇员名称,敲入如下命令:\\
awk '\$3 == 0 \{ print \$1  \}'  emp.data \\
此处pattern --\$3 == 0,匹配每行数据的第三个字段是否等于0,对应的action--\{ print \$1  \},打印匹配行的第一个字段。\\

当你读这本书的时候,最好试着去执行和修改上述的程序。因为大部分程序都很短,所以你能够很快的明白awk工作的原来。在Unix系统上,上述的两个例子在终端显示如下:\\
\$awk  '\$3 > 0 \{ print \$1 , \$2 * \$3  \} ' emp.data\\
\indent Kathy   40 \\
\indent Mark    100 \\
\indent Mary    110 \\
\indent Susie  76.5 \\
\indent \$awk '\$3 == 0 \{ print \$1  \} emp.data \\
\indent Beth \\
\indent Dan  \\
\indent \$   \\
\indent 每行的起始符号\$是Unix系统的输入提示符;它可能与你的系统有所区别。 \\

\title{\textbf{Awk程序的结构}} \\
让我们退后一步看看到底发生了什么事情。上面命令行中的单引号部分是用awk语言写的程序。在这一章出现的所以awk程序都是由单条或者多条patter-action语句组成的:\\
pattern    \{ action  \}  \\
pattern    \{ action  \}  \\
... \\
awk程序最基本的操作是逐条扫描输入行,以搜索到那些与pattern相匹配的行。在讨论中,”匹配(match)”一词的确切意思依赖于pattern;例如:如果pattern是\$3>0,那么它的意思是“此条件为真”。 \\
所有的pattern都会对输入的每行数据依次进行测试。只要有一个pattern匹配上,则相关的action就会被执行。然后,读入下一行,重新开始匹配。直到处理完所有的行,程序才结束。 \\
以上的程序是典型的pattern和action的例子。 \\
\$3 == 0 \{print\$1\} 是单个pattern-action语句;只要哪行的第三字段等于0,那么就打印该行的第一字段。 \\
可以省略pattern-action语句中的pattern或者action(但不同时)。如果pattern没有对应的action,例如:\$3 == 0 ,那么匹配的行(即,\$3==0返回真的行)都会被打印出来。当输入文件是emp.data时,该程序会打印第三字段为0的两行: \\
\indent Beth   4.00    0 \\
\indent Dan    3,75    0 \\
如果action没有对应的pattern,例如:\{print\$1 \},那么该action会每行数据的第一个字段。 \\
因为pattern和action都是可以选择的,所以action用花括号括起来,以与pattern相区别。 \\
\indent \title{\textbf{执行一个awk程序}} \\
有许多方式来执行一个awk程序。你可以在命令行下输入awk  'program' input files来对买个输入文件执行'program'。例如,你可以输入 \\
awk  '\$3 == 0 \{ print \$1 \} '  file1  file2  \\
来打印文件file1和file2中第三个字段是0的所有行。\\
你可以在命令行中省略输入文件,而仅仅输入 \\
awk  'program' \\
在这种情况下,awk会对你在终端的任何输入执行program,直到你键入文件结束信号(ctr-d在Unix系统下)。下面是在Unix系统下的一个简单会话: \\
\$ awk  ' \$3 == 0 \{  print  \$1  \} '  \\
Beth      4.00      0 \\
Beth  \\
Dan       3.75      0 \\
Dan   \\ 
Kathy     3.75      10 \\
Kathy     3.75      0 \\
Kathy \\
...   \\
加粗的字符串是计算机打印的。 \\
此种方式很容易用awk做实验:输入你的程序,然后输入数据,接着查看执行结果。我们再次鼓励你尝试输入这些例子,也可以尝试修改例子。 \\
注意在命令行中程序要用单引号引起来。这样做是为了防止程序中的字符\$被shell解释,并且可以允许程序超过单行限制。 \\
当程序很短时(只有几行),此种执行方式很方便。但是,当程序很长的时候,把程序写入单独的文件(如:progfile),则是更方便的做法。输入如下命令行: \\
awk  -f  progfile  optional  list   of  input  files \\
参数-f告诉awk根据文件名获取程序。可以用任何文件名称来代替progfile。 \\
\indent\title{\textbf{错误}} \\
如果你在awk程序中犯了一个错误,那么awk将输出一个诊断信息。例如,如果你错误输入括号,如下:\\
awk   '\$3 == 0 \{  print \$1 \} '  emp.data \\
你讲得到如下提示信息: \\
awk : syntax  error  at  source  line 1 \\
   context   is \\
   \$3 == 0 \ \ \ \emph{$\ggg$} [  \emph{$\lll$} \\
      extra  \}  \\
	  missing \} \\
   awk :bailing  out at source  line 1 \\
   \indent"Syntax error“ 意味着你在”\emph{$\ggg$} \emph{$\lll$}“标注的地方犯了一个语法错误。”Bailing out“意味着恢复尝试失败。有时候,你可能获得一个更有用的信息,比如错误的大括号或圆括号匹配信息。 \\
\indent 由于语法错误,awk将不会去尝试执行程序。然而,有些错误只有在运行的时候才可能被检测出来。例如,如果你尝试用0除一个数,那么akw将立即停止程序,并报告输入的行号和程序错误的位置。 \\

\subsection{简单输出} 
剩下的章节包含许多短小经典的awk程序,主要用于处理以上章节提到的emp.data文件。我们将简短的介绍接下来的内容,但是这些例子同时也说明了awk很擅长这方面的操作--打印字段,过滤输入和转换数据。我们不打算展示awk所有的功能,但是我们会针对特殊情况进行细节上的讨论。学完本章,你将会完成相当多的例题,最后你会发现能够更容易的理解接下来的内容。\\
\indent 我们将会只展示程序部分,而不是整个命令行。无论是把程序用单引号引起来作为命令行的一部分执行,还是把程序写进文件再以-f参数的形式调用,这些程序都能够正常运行。 \\
\indent awk语言仅仅只有两种数据类型:数字和字符串。emp.data文件具有这两种类型的典型特征--字符串组合和被空格或制表符分隔的数字。
\\
\indent Awk每次只读入输入内容的一行数据,同时把该行分隔成多个字段,默认情况下,每个字段是一个不含空格 或制表符的字符序列。当前行的第一个字段用\$1表示,第二个字段用\$2表示,依次类推。整行数据用\$0表示。字段标号根据具体行改变。 \\
我们通常需要输出每行的部分或者所有字段,或许会进行一些计算。本部分的程序基本都是都是这种样式。 \\

\title{\textbf{打印所有行}}  \\
\indent 如果action没有对应的pattern,那么action会对所有的行进行处理。语句print将打印当前所有输入行,所以程序\{print\}打印标准输入的所有行。因为\$0代表整行,所以\{ print \$0  \}与\{ print  \}操作相同。 \\

\title{\textbf{打印指定字段}} \\
\indent 使用单条语句可以在同一输出行打印多个字段。程序\{print \$1,\$3\}可以打印每一输入行的第一和第三个字段。当输入文件是emp.data时,输入如下: \\
\indent Beth     0 \\
\indent Dan      0 \\
\indent Kathy   10 \\
\indent Mark    20 \\
\indent Mary    22 \\
\indent Susie    18 \\
\indent 打印语句中被逗号分隔的表达式在输出过程中默认用空格代替。print会在所有生成语句的末端添加换行符。以上两种默认操作都可以被修改。这些我们将在第二章展示。 \\









\end{document}
