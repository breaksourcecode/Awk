\documentclass[10pt,fancyhdr,UTF-8]{ctexart}
\setmainfont{Times New Roman}
\setCJKmainfont{SimSun}
\setCJKfamilyfont{song}{STSong}
\newcommand\zh[1]{{\CJKfamily{song}#1}}
%\usepackage[centering,paperwidth=180mm,paperheight=230mm,body={390pt,530pt},showframe]{geometry}
\usepackage[centering,body={390pt,530pt},showframe]{geometry}
%\usepackage[top=0in,bottom=10in,left=1.25in,right=1.25in]{geometry}
\geometry{left=2.5cm,right=2.5cm,bottom=12.5cm,top=0cm}
\usepackage[a4,center,frame,color=blue]{crop}
\usepackage{indentfirst} %段缩进
\setlength{\parindent}{2em}
\usepackage{titlesec}
\usepackage{CJK}
%\usepackage{titling}
%\pretitle{\begin{flushright}\large}
\usepackage[flushleft]{caption}
\usepackage{fancyhdr}
\pagestyle{fancy}
\rfoot{\thepage}
%\fancyhf{}
%\fancyhead[RE]{\normalfont\small\rmfamily\nouppercase{\leftmark}}
%\fancyhead[LO]{\normalfont\small\rmfamily\nouppercase{\rightmark}}
%\fancyhead[LE,RO]{\thepage}
%\fancyhead[LE,LO]{\small Book Title}
\topmargin=0pt
\oddsidemargin=0pt \evensidemargin=0pt
\textwidth=14cm
\textheight=21cm
\title{AWK教程}
\author{half-beast@163.com}
\date{Jun. 1 , 2015}
\begin{document}
\maketitle
\section{AWK教程}
\indent Awk是一门方便其表达力丰富的编程语言,它适用于多变量计算和数据操作任务。本章是一个教程,目标是使你能够尽快的用awk写出属于自己的程序。第二章详细描述了语言的方方面面,而剩下的章节则向你展示awk习惯于解决许多领域的问题。整本书中,我们都试图挑选一些让你感觉既有意思同时又很有启发性的例子。\par
\subsection{开始}
\indent 有用的awk程序通常都很短,有的甚至只有一到两行。假设你有一份称为emp.data的文件。它包含雇员名字,时新和雇员的工作时常,一条雇员记录单独成行,如下所示:\\
\indent Beth 4.00 0\\
\indent Dan  3.75 0\\
\indent Kathy 4.00 10\\
\indent Mark 5.00 20\\
\indent Mary 5.50 22\\
\indent Susie 4.25 18\\
现在你想打印所有那些工作时间超过零小时的雇员名称和个人所得。awk天生擅长这类工作,对awk来说,这太容易了。只需敲入以下命令行:\\
awk `\$3 \$>\$ 0 \{ print \$1 , \$2  *  \$3  \}' emp.data \\
你应该得到如下输出:\\
\indent Kathy   40\\
\indent Mark    100\\
\indent Mary    110\\
\indent Susie   76.5\\
\indent 这条命令告诉系统使用awk来执行单引号内的程序,并从输入文件emp.data中获取处理数据。单引号中的内容是一条完整的awk程序。它由单条pattern-action语句构成。\\
pattern--\$3 > 0,用于匹配每一输入行的第三列或字段是否大于0,对应的action--\{ print \$1 , \$2 * \$3  \}打印匹配行的第一列和第二列与第三列的乘积。\\
如果你想打印那些没有工作的雇员名称,敲入如下命令:\\
awk '\$3 == 0 \{ print \$1  \}'  emp.data \\
此处pattern --\$3 == 0,匹配每行数据的第三个字段是否等于0,对应的action--\{ print \$1  \},打印匹配行的第一个字段。\\

当你读这本书的时候,最好试着去执行和修改上述的程序。因为大部分程序都很短,所以你能够很快的明白awk工作的原来。在Unix系统上,上述的两个例子在终端显示如下:\\
\$awk  '\$3 > 0 \{ print \$1 , \$2 * \$3  \} ' emp.data\\
\indent Kathy   40 \\
\indent Mark    100 \\
\indent Mary    110 \\
\indent Susie  76.5 \\
\indent \$awk '\$3 == 0 \{ print \$1  \} emp.data \\
\indent Beth \\
\indent Dan  \\
\indent \$   \\
\indent 每行的起始符号\$是Unix系统的输入提示符;它可能与你的系统有所区别。 \\

\title{\textbf{Awk程序的结构}} \\
让我们退后一步看看到底发生了什么事情。上面命令行中的单引号部分是用awk语言写的程序。在这一章出现的所以awk程序都是由单条或者多条patter-action语句组成的:\\
pattern    \{ action  \}  \\
pattern    \{ action  \}  \\
... \\
awk程序最基本的操作是逐条扫描输入行,以搜索到那些与pattern相匹配的行。在讨论中,”匹配(match)”一词的确切意思依赖于pattern;例如:如果pattern是\$3>0,那么它的意思是“此条件为真”。 \\
所有的pattern都会对输入的每行数据依次进行测试。只要有一个pattern匹配上,则相关的action就会被执行。然后,读入下一行,重新开始匹配。直到处理完所有的行,程序才结束。 \\
以上的程序是典型的pattern和action的例子。 \\
\$3 == 0 \{print\$1\} 是单个pattern-action语句;只要哪行的第三字段等于0,那么就打印该行的第一字段。 \\
可以省略pattern-action语句中的pattern或者action(但不同时)。如果pattern没有对应的action,例如:\$3 == 0 ,那么匹配的行(即,\$3==0返回真的行)都会被打印出来。当输入文件是emp.data时,该程序会打印第三字段为0的两行: \\
\indent Beth   4.00    0 \\
\indent Dan    3,75    0 \\
如果action没有对应的pattern,例如:\{print\$1 \},那么该action会每行数据的第一个字段。 \\
因为pattern和action都是可以选择的,所以action用花括号括起来,以与pattern相区别。 \\
\indent \title{\textbf{执行一个awk程序}} \\
有许多方式来执行一个awk程序。你可以在命令行下输入awk  'program' input files来对买个输入文件执行'program'。例如,你可以输入 \\
awk  '\$3 == 0 \{ print \$1 \} '  file1  file2  \\
来打印文件file1和file2中第三个字段是0的所有行。\\
你可以在命令行中省略输入文件,而仅仅输入 \\
awk  'program' \\
在这种情况下,awk会对你在终端的任何输入执行program,直到你键入文件结束信号(ctr-d在Unix系统下)。下面是在Unix系统下的一个简单会话: \\










\end{document}
